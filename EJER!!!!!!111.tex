\documentclass[a4paper,12pt]{article}
\usepackage[spanish]{babel}
\usepackage{tikz}
\usetikzlibrary{arrows.meta, positioning}

\begin{document}

\begin{center}
\textbf{\Large Diagrama de Clases – Entidades Geográficas}
\end{center}

\vspace{1cm}

\begin{tikzpicture}[
    class/.style={
        rectangle,
        draw,
        rounded corners,
        minimum width=4cm,
        minimum height=1cm,
        align=center
    },
    relation/.style={
        -{Triangle[length=3mm]},
        thick
    },
    aggregation/.style={
        o-,
        thick
    }
]

% Clases
\node[class] (entidad) {
    \textbf{EntidadGeográfica} \\ \hline
    nombre : String \\
    código : String
};

\node[class, below left=2cm and 3cm of entidad] (punto) {
    \textbf{Punto} \\ \hline
    x : double \\
    y : double
};

\node[class, below=2cm of entidad] (linea) {
    \textbf{Línea}
};

\node[class, below right=2cm and 3cm of entidad] (area) {
    \textbf{Área}
};

% Herencia
\draw[relation] (punto.north) -- (entidad.south);
\draw[relation] (linea.north) -- (entidad.south);
\draw[relation] (area.north) -- (entidad.south);

% Relaciones
\draw[aggregation] (linea.west) -- node[above]{2..*} (punto.east);
\draw[aggregation] (area.east) -- node[above]{3..*} (punto.west);

\end{tikzpicture}

\end{document}
